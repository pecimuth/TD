\documentclass[12pt]{article}
\usepackage[utf8]{inputenc}
\usepackage{geometry}
\geometry{
    a4paper,
    total={170mm,257mm},
    left=20mm,
    top=20mm
}

\title{Tower Defense}
\author{Andrej Pečimúth}
\date{Programování v C++, ZS 2019/2020}

\begin{document}
\maketitle

\section{Používateľská príručka}

\subsection{Inštalácia}
Najprv potrebujeme zostaviť knižnicu SFML. Predpokladá sa operačný systém Windows 10
a Visual Studio 2019. Hra by ale mala byť zostaviteľná na každej platforme podporovanej
knižnicou SFML.

\subsubsection{Zostavenie SFML}

\begin{enumerate}
    \item Zložku SFML-2.5.x otvorte vo Visual Studiu.
    \item Vo Visual Studiu kliknite na CMakeLists.txt.
    \item Zvoľte konfiguráciu, napríklad x64-Debug, a stlačte F5.
\end{enumerate}

\subsubsection{Zostavenie TD}

\begin{enumerate}
    \item Otvorte súbor TD.sln vo Visual Studiu.
    \item Otvorte Properties pre projekt TD.

        V sekcií C++/General editujte Additional Include Directiories, aby zahŕňal
        hlavičkové súbory SFML podľa vzoru nižšie. Prefix \verb|C:\path\to\repo| nahraďte
        skutočnou cestou k SFML. Vzor: \\
        \verb|C:\path\to\repo\SFML-2.5.x\include%(AdditionalIncludeDirectories)}|


        V sekcií Linker/General editujte položku Additional Library. Časť \verb|x64-Debug|
        nahraďte požadovanou konfiguráciou, ktorú ste zvolili pri zostavení SFML.
        Vzor: \\
        \verb|C:\path\to\repo\SFML-2.5.x\out\build\x64-Debug\lib;| \\
        \verb|%(AdditionalLibraryDirectories)|


        V sekcií Debugging editujte položku Environment, aby mal program prístup
        k potrebným .dll súborom. 
        Vzor: \\
        \verb|PATH=\%PATH\%;C:\path\to\repo\SFML-2.5.x\out\build\x64-Debug\lib;| \\
        \verb|C:\path\to\repo\SFML-2.5.x\extlibs\bin\x6\$(LocalDebuggerEnvironment)|
    \item Zvoľte konfiguráciu a stlačte F5.
\end{enumerate}

\section{Programátorská príručka}

\subsection{Vstupný bod a smyčka}
Hlavnú úlohu zohráva trieda \emph{App}. Jej metóda \emph{load} je zavolaná práve raz vo vstupnom
bode programu. Táto metóda sa postará o načítanie všetkých textúr, fontov a zvukov.
Následne metóda loop spustí hernú smyčku (game loop). Trieda \emph{App} pracuje so scénami.
Scéna je analógia jednej stránky vrámci webstránky. \emph{App} vlastní najviac jednu inštanciu
scény a rieši prechody medzi scénami, keď si to scéna vyžiada.

Game loop sa opakuje kým je okno aplikácie otvorené. Pozostáva z nasledujúcich častí:

\begin{enumerate}
    \item Prechod medzi scénami, ak si to práve aktívna scéna vyžiadala. \emph{App} predá
    konštruktoru novej scény parametre z objektu \emph{SceneChangeRequest}.
    \item Spracovanie vstupu - eventov. Napríklad kliky myšou, klávesnica apod.
    Pre každú udalosť - \emph{sf::Event} sa volá metóda \emph{Scene::handleInput(event)}.
    

    Toto je v aplikácií zaužívaný princíp. Objekty, ktoré vlastnia iné objekty,
    na nich volajú metódu \emph{handleInput(event)} pre každý obdržaný event.
    \item Niekoľko krokov v hernom svete \emph{fixedTimestepUpdate}, používa sa \emph{fixed timestep update},
    To znamená, že v každom cykle sa spraví toľko krokov v hernom svete, aby sa za každú sekundu spravilo 30 krokov.
    Scéna spraví krok po zavolaní \emph{update(delta)}. \emph{Delta} je časová dĺžka,
    o ktorý sa má svet pohnút dopredu. Keďže robím 30 krokov za sekundu, je to $\frac{1}{30}$ sekundy.
    
    
    Tento princíp sa zase opakuje naprieč celou aplikáciou. Rôzne objekty vo svojej metóde
    \emph{update} volajú \emph{update(delta)} na objekty, ktoré vlastnia.
    \item Kreslenie scény. Kresliteľné objekty dedia z triedy \emph{sf::Drawable} a implementujú
    jej virtuálnu metódu \emph{draw(target, states)} v ktorej vykreslia seba a objekty, ktoré vlastnia.
\end{enumerate}

\subsection{Scény}
Implementované sú nasledujúce scény:
\begin{itemize}
    \item $WelcomeScene$ je 
\end{itemize}

\end{document}
